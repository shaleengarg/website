\documentclass[10pt]{article}
\usepackage{fullpage}
\usepackage{amsmath}
\usepackage{hyperref}
\usepackage{amssymb}
\usepackage{url}
\usepackage{multicol}
\usepackage[usenames]{color}
\usepackage{enumitem}
\usepackage{nopageno}
\usepackage{ragged2e}

\renewcommand{\arraystretch}{1.5}
\setlist{nolistsep}
\leftmargin=0.10in %% was .25in
\oddsidemargin=0.25in
\textwidth=6.0in
\topmargin=-0.75in
\textheight=10.50in

\raggedright
\pagestyle{empty}
%\pagenumbering{arabic}

\def\bull{\vrule height 0.8ex width .7ex depth -.1ex }

\newenvironment{changemargin}[2]{%
    \begin{list}{}{%
            \setlength{\topsep}{0pt}%
            \setlength{\leftmargin}{#1}%
            \setlength{\rightmargin}{#2}%
            \setlength{\listparindent}{\parindent}%
        \setlength{\itemindent}{\parindent}%
            \setlength{\parsep}{\parskip}%
        }%
\item[]}{\end{list}
}

\newcommand{\lineover}{
    \begin{changemargin}{-0.05in}{-0.10in}
        \vspace*{-8pt}
        \hrulefill \\
        \vspace*{-2pt}
    \end{changemargin}
}

\newcommand{\header}[1]{
    \begin{changemargin}{-0.75in}{-0.75in}
        \scshape{#1}\\
        \lineover
    \end{changemargin}
}

\newcommand{\name}[1]{
    \begin{changemargin}{-0.6in}{-0.6in}
        \begin{center}
            {\Large \scshape {#1}}
        \end{center}
    \end{changemargin}
}

\newcommand{\contact}[6]{
    \begin{changemargin}{-0.65in}{-0.65in}
        \begin{multicols}{2}
            {#1}\\ \smallskip 
            {#2}\\ \smallskip
            {#3}\\ \smallskip
            {#4}\\ \smallskip 
            {#5}\\ \smallskip
            {#6}
        \end{multicols}
    \end{changemargin}
}

\newenvironment{body} {
    \vspace*{-16pt}
    \begin{changemargin}{-0.6in}{-0.65in}
    }   
    {\end{changemargin}
}   

\newcommand{\school}[4]{
    \textbf{#1} \hfill \emph{#2\\}
    #3\\ 
    #4\\
}

% END RESUME DEFINITIONS

\begin{document}

%%%%%%%%%%%%%%%%%%%%%%%%%%%%%%%%%%%%%%%%%%%%%%%%%%%%%%%%%%%%%%%%%%%%%%%%%%%%%%%%
% Name
\name{Shaleen Garg}

\contact{Fifth Year Dual Degree Student}{Computer Science and Engineering}{International Institute of Information Technology, Hyderabad}{\hspace{60pt} \url{http://researchweb.iiit.ac.in/~shaleen.garg}}{\hspace{60pt} \textbf{Email:} shaleen.garg@research.iiit.ac.in}{\hspace{60pt} }

%Research Interests
\header{Research Interests}
\vspace{14pt}
\begin{body}
    Parallel \& Distributed Algorithms, IoT, Cloud Computing, Computer Architecture and Compilers.
\end{body}
\smallskip

%%%%%%%%%%%%%%%%%%%%%%%%%%%%%%%%%%%%%%%%%%%%%%%%%%%%%%%%%%%%%%%%%%%%%%%%%%%%%%%%
% Education
\header{Education}
\vspace{14pt}
\begin{body}
    {\bf Masters by Research in Computer Science} \hfill {August 2018 - May 2019(expected)} 
    \\ International Institute of Information Technology, Hyderabad,
    GPA: 9.25/10
\end{body}
\vspace{16pt}
\begin{body}
{\bf Bachelor of Technology in Computer Science} \hfill {August 2014 - May 2018} 
    \\ International Institute of Information Technology, Hyderabad, GPA: 7.06/10
\end{body}
\vspace{16pt}
\begin{body}
    {\bf AISSE(CBSE) Class XII} \hfill {August 2013 - May 2014} 
    \\ FAIPS(DPS), Kuwait, 91.2\%
\end{body}
%\vspace{2pt}
%\begin{center}
%\begin{tabular}{|l|l|l|l|}
%    \hline 
%    \textbf{Year} & \textbf{Degree} & \textbf{Institute} & \textbf{GPA} \\ \hline
%    2019 (Expected) & CSE (BTech + MS) & IIIT, Hyderabad & 7.3/10.0 (8 semesters) \\ \hline
%    2014 & AISSE, XII (CBSE) & FAIPS (DPS), Kuwait & 91.2\% \\ \hline
%    2012 & CISCE, X (ICSE) & The Aryan School, Dehradun  & 87\% \\ \hline
%\end{tabular}
%\end{center}
%\smallskip
%%%%%%%%%%%%%%%%%%%%%%%%%%%%%%%%%%%%%%%%%%%%%%%%%%%%%%%%%%%%%%%%%%%%%%%%%%%%%%5
\smallskip
\header{Research Summary}
\begin{body}
    \vspace{14pt}
    \justifying
    A lot of co-processors and accelerators are being used across different fields these days for example algorithmic trading, crypto-currency mining, deep learning, computational natural sciences etc. Research in these fields has burgeoned many folds; but most accelerators give a limited API level control to the user and do not expose an operating system. They also do not support time sharing.
    \\
    This inability limits their applicability especially in environments such as Platform-as-a-Service (PaaS) and Resource-as-a-Service (RaaS). In the former, elastic demands may require preemption where as in the latter, fine-grained economic models of service cost can be supported with time sharing. Moreover, the demand for these accelerators has far exceeded their supply making them very costly; inaccessible to people in resource constrained environments.
    \\
    As a part of my research, we have created a deployable system which, with minimal programmer support, efficiently emulates preemption on accelerators and also provide memory guarantees to the workloads contending for accelerator time. This system aims to provide a software abstraction for all the underlying heterogeneous accelerators so that workloads can be created and executed agnostic to specific hardware.
    \\
    This will reduce our dependence on a specific accelerator, enable elastic allocation of accelerators for PaaS environments, enable automatic checkpointing of workloads, enable elastic time sharing for RaaS environments and most importantly reduce the cases of starvation experienced by small processes contending for accelerator time.
\end{body}
%%%%%%%%%%%%%%%%%%%%%%%%%%%%%%%%%%%%%%%%%%%%%%%%%%%%%%%%%%%%%%%%%%%%%%%%%%%%%%%%
% Research Projects and Works
\smallskip
\header{Publications}

\begin{body}
    \vspace{14pt}
    \textbf{Share-a-GPU: Providing Simple and Effective Time-Sharing on GPUs}\\
    \emph{Shaleen Garg, Kishore Kothapalli, Suresh Purini}\\
    25th IEEE International Conference on High Performance Computing, Data, and Analytics (HiPC), Bengaluru, India, December 17-20, 2018.
    \smallskip 
\end{body}

\begin{body}
    \vspace{14pt}
    \textbf{GPUScheduler: User Level Preemptive Scheduling for NVIDIA GPUs}\\
    \emph{Shaleen Garg, Kishore Kothapalli, Suresh Purini}\\
    24th IEEE International Conference on High Performance Computing, Data, and Analytics (HiPC-SRS10), Jaipur, India, December 18-21, 2017. (Student Research Poster)
\end{body}

%%%%%%%%%%%%%%%%%%%%%%%%%%%%%%%%%%%%%%%%%%%%%%%%%%%%%%%%%%%%%%%%%%%%%%%%%%%%%%
\smallskip
\header{Honours And Awards}
\vspace{14pt}
\begin{body}
    Dean's Undergraduate Research Award, IIIT-H  \hfill {AY '17 - '18}
\end{body}
\vspace{14pt}
% \begin{body}
%     Student Travel Grant, PACT  \hfill {2017}
% \end{body}

%%%%%%%%%%%%%%%%%%%%%%%%%%%%%%%%%%%%%%%%%%%%%%%%%%%%%%%%%%%%%%%%%%%%%%%%%%%%%%%%
% Research Projects and Works
\smallskip
\header{Current Research Projects}

 \begin{body}
     \vspace{14pt}
     \textbf{Expedited Results using Online Prioritization of GPU Kernels} \hfill \emph{September'18 - Ongoing}\\
     \emph{Dr.~Kishore Kothapalli, Dr.~Suresh Purini \& Dr.~Pawan Kumar}
     \begin{itemize}
         %\item{Showing that Dynamically changing the priority of different parts of a kernel on GPU gives a coarse grain result very efficiently}
         \item{Automatically changing priorities of different parts of a kernel on GPU to expedite coarse grain results.}
         %\item{Priorities depend on the rate of change of values in gradient descent/ascent.}
         \item{Concurrently calculate the saturation condition on CPU to prioritize kernel without additional overheads on GPU.}
     \end{itemize}
     \smallskip 
 \end{body}

% \begin{body}
%     \vspace{14pt}
%     \textbf{Optimizing Floating Point Numbers and their Arithmetic for FPGAs} \hfill \emph{August'18 - Ongoing}\\
%     \emph{Dr.Kishore Kothapalli \& Dr.Srikanth Sridharan}
%     \begin{itemize}
%         \item{Exploring and modifying Unums/Posits for optimizing floating pointing representation.}
%         \item{Trying to come up with alternate ways for representing floating point numbers with efficient arithmetic logic.}
%         \item{Aiming to optimize bit representation, energy impact and implementation area for DSP applications on FPGAs.}
%     \end{itemize}
%     \smallskip 
% \end{body}

\begin{body}
    \vspace{14pt}
    \textbf{Reinforcement Learning for Financial Portfolio Management} \hfill \emph{August'18 - Ongoing}\\
    \emph{Dr.~Pawan Kumar}
    \begin{itemize}
        \item{Reducing the portfolio management problem to a typical game(MDP).}
        \item{Design the agent to predict the price of a commodity after some time using SARSA.}
        \item{Let the agent master the game on historical trading data by maximizing the portfolio value.}
        \item{Allow the agent to explore by changing the commodities in the portfolio.}
    \end{itemize}
    \smallskip 
\end{body}

% \begin{body}
%     \vspace{14pt}
%     \textbf{Preemptive Scheduling on NVIDIA GPUs} \hfill \emph{September'16 - July'18}\\
%     \emph{Dr.Kishore Kothapalli(Associate Prof, IIIT-H) \& Dr.Suresh Purini(Assistant Prof, IIIT-H)}
%     \begin{itemize}
%         \item{Developed a proof-of-concept scheduler using C and CUDA API which simulates preemptive scheduling and simulates time slicing for general purpose CUDA programs on NVIDIA GPUs.}
%         \item Used concepts gained in the above scheduler to provide a generic software-level preemptive scheduler for general purpose single-GPU applications.
%         \item Provided a software memory manager for all the resident programs in order to provide GPU global memory gurantees.
%         \item Load balancing for multi-GPU setting involving the above concepts.
%     \end{itemize}
%     \smallskip 
% \end{body}

%%%%%%%%%%%%%%%%%%%%%%%%%%%%%%%%%%%%%%%%%%%%%%%%%%%%%%%%%%%%%%%%%%%%%%%%%%%%%%%%
% Major Projects and Works
\newpage
 \header{Relevant Projects}

 \begin{body}
     \vspace{14pt}
%     % \textbf{Optimizing Reinforcement Learning for Financial Portfolio Management} \hfill \emph{August'18 - Ongoing}\\
%     % \emph{Independent Project: Dr.Pawan Kumar (Assistant Prof, IIIT-H)}
%     % \begin{itemize}
%     %     \item{Trying to find the bottlenecks in these kind of workloads and}
%     %     \item{Optimizing the computations using mathematical tricks from supercomputing such as matrix optimizations}
%     % \end{itemize}
%     % \smallskip
    
%     % \textbf{Optimizing Floating Point Numbers and their Arithmetic for FPGAs} \hfill \emph{August'18 - Ongoing}\\
%     % \emph{Dr.Kishore Kothapalli(Associate Prof, IIIT-H) \& Dr.Srikanth Sridharan(Senior Developer, HPC@Applied Materials)}
%     % \begin{itemize}
%     %     \item{Exploring Unums/Posits for optimizing floating pointing representation}
%     %     \item{Trying to come up with alternate ways for representing floating point numbers with efficient arithmetic logic}
%     %     \item{Aiming to optimize bit representation, energy impact and implementation area for DSP applications on FPGAs}
%     % \end{itemize}
%      \smallskip

    \href{https://researchweb.iiit.ac.in/~shaleen.garg/files/Perf_engg_project.pdf}{\textbf{Performance Engineering of Wireless IoT Sensors}} \hfill \emph{November'16 - December'16}\\
    \emph{Course Project: Dr.~Anil Gurijala}
    \begin{itemize}
        \item{Designed a low-cost Wireless IoT Boilerplate hardware platform using readily available 8-Bit micro-controller (arduino micro) and other hardware modules like ESP 8266.}
        \item{The hardware is capable of housing 5-10 sensors (depending on the pins available on the arduino model).}
        \item{Tested the designed hardware platform to maximize both data transmission frequency and up-time (of the hardware), when connected to a finite remote power source like commodity ``AA'' batteries.}
        \item{Was successful in keeping the cost of the whole system as low as \$22.}
    \end{itemize}
     \smallskip

     \href{https://github.com/mayukuse24/HDFS}{\textbf{Distributed Grep}} (Team-Mate \href{https://github.com/mayukuse24}{Mr.Vinaya Khandelwal}) \hfill \emph{October'16 - November'16}\\
     \emph{Course Project: Dr.Vivekananda Vellanki}\\
     \begin{itemize}
         \item{Implemented Hadoop Distributed File System(HDFS) in Java. Took care of underlying failures associated with these systems.}
         \item{Implemented generic Map-Reduce program over the HDFS and tested distributed grep on it.}
     \end{itemize}
     \smallskip
  
     \textbf{Bflat Programming Language} \hfill \emph{August'17 - November'17}\\
     \emph{Course Project: Dr.Suresh Purini}
     \begin{itemize}
         \item{Implemented a interpreter for Bflat(self-defined) language using C++.}
         \item{Implemented front-end compiler for LLVM Intermediate Representation generation.}
         \item{Used Flex(for tokens) and Bison(for grammar).}
     \end{itemize}
     \smallskip

     \href{https://researchweb.iiit.ac.in/~shaleen.garg/files/Summer2.pdf}{\textbf{Distributed Graph Algorithms}} \hfill \emph{May'16 - July'16}\\
     \emph{Summer Project: Dr.Govindarajulu R}
     \begin{itemize}
         \item {Implemented ``Asynchronous concurrent-initiator depth first search spanning tree'' and ``Synchronous Breadth First Spanning Tree'' using Erlang}
         \item {Tested the above algorithms on 15 nodes with graphs of size as large as 50 million vertices.}
     \end{itemize}
     \smallskip
\end{body}

%%%%%%%%%%%%%%%%%%%%%%%%%%%%%%%%%%%%%%%%%%%%%%%%%%%%%%%%%%%%%%%%%%%%%%%%%%%%%%%%
%Teaching Experience
\header{Teaching and Research Experience}
\vspace{14pt}
\begin{body}
    \textbf{Teaching Assistant} \hfill{August'17 - May'18}
    \\ \emph{Course: Distributed Systems (CSE431)}
    \begin{itemize}
        \item Prepared class activities focusing on the real work applications of distributed systems for $\sim$100 senior level undergraduate and postgraduate students.
        \item Created and graded course assignments to ensure students understood material and stayed on track.
    \end{itemize}
\end{body}
\vspace{14pt}
\begin{body}
    \textbf{Graduate Research Assistant} \hfill{AY '18 - '19}
    \\ \emph{Centre for Security Theory and Algorithmic Research(CSTAR)}
        \begin{itemize}
          \item{Modifying Unums/Posits for optimizing floating pointing representation.}
          \item{Trying to come up with alternate ways for representing floating point numbers with efficient arithmetic logic.}
          \item{Aiming to optimize bit representation, energy impact and implementation area of floating points for DSP applications on FPGAs.}
        \end{itemize}
\end{body}
\smallskip

%%%%%%%%%%%%%%%%%%%%%%%%%%%%%%%%%%%%%%%%%%%%%%%%%%%%%%%%%%%%%%%%%%%%%%%%%%%%%%%%
\header{University And Community Service}
\vspace{14pt}
\begin{body}
    Volunteer - Gave weekly science lessons to local underprivilged children at Ashakiran (IIIT-H initiative) \hfill{AY '16 - '17}
\end{body}
\vspace{14pt}
\begin{body}
    Student System Administrator - Maintained all the compute and storage nodes at CSTAR. \hfill{AY '17 - '19}
\end{body}
\vspace{14pt}
\begin{body}
    Student's Parliament \hfill{AY '18 - '19}
    \begin{itemize}
        \item Elected Member of the student's parliament representing the Masters Students at IIIT Hyderabad.
        \item Chief Election Commissioner for Student's Parliament Elections.
        \item Student Member of the Disciplinary Sub-Committee(DISCO) at IIIT Hyderabad.
    \end{itemize}
\end{body}



%%%%%%%%%%%%%%%%%%%%%%%%%%%%%%%%%%%%%%%%%%%%%%%%%%%%%%%%%%%%%%%%%%%%%%%%%%%%%%%%
% Short Term Projects and Hacks
% \header{Short Term Projects}

% \begin{body}
%     \vspace{14pt}

%     % \href{https://github.com/shaleengarg/UltimateTic-Tac-Toe-Bot}{\textbf{Ultimate Tic-Tac-Toe Bot}} \hfill \emph{February'16}\\
%     % \emph{Course Project: Dr.Praveen Paruchuri (Associate Prof, IIIT-H), Artificial Intelligence}
%     % \begin{itemize}
%     %     \item{A python bot to play 9x9 ultimate tic-tac-toe. The bot uses 5 ply deep Alpha-Beta pruning to evaluate the next move based on self developed heuristics to win.}
%     % \end{itemize}
%     % \smallskip

%     % \href{https://github.com/shaleengarg/python2Donkeykong}{\textbf{DonkeyKong}} \hfill \emph{September'15}\\
%     % \emph{Course Project: Dr.Raghu Reddy(Associate Prof, IIIT-H), Structured Systems Analysis and Design (SSAD)}
%     % \begin{itemize}
%     %     \item {An ASCII implementation of the classical Donkey Kong game}
%     %     \item {It showcases and rigorously uses all the concepts of Object Oriented Programming in python}
%     % \end{itemize}
%     % \smallskip

%     \href{https://github.com/shaleengarg/C-shell}{\textbf{C-Shell}} \hfill \emph{October'15}\\
%     \emph{Course Project: Dr.Suresh Purini(Assistant Prof, IIIT-H), Operating Systems}
%     \begin{itemize}
%         \item {A C implementation of the shell using commands like execvp, fork, signal et cetera}
%         \item {The shell has capabilities like piping, redirection and multiple commands}
%     \end{itemize}
%     \smallskip


%     % \href{https://youtu.be/_XdZ6j8wHjg}{\textbf{Phone Wars}} (Team-Mates \href{https://github.com/Punyaslok}{Mr.Punyaslok Pattnaik} and Mr.Aagam Shah) \hfill \emph{April'16}\\
%     % \emph{Course Project: Dr.PJ Narayanan(Professor \& Director, IIIT-H), Graphics}
%     % \begin{itemize}
%     %     \item {Produced an animated movie to showcase all the different concepts acquired in the course including camera positions, lighting et cetera.}
%     % \end{itemize}
%     % \smallskip

%     % \href{https://researchweb.iiit.ac.in/~shaleen.garg/Carrom/}{\textbf{Online Carrom Game}} \hfill \emph{March'16}\\
%     % \emph{Course Project: Dr.PJ Narayanan(Professor \& Director, IIIT-H), Graphics}
%     % \begin{itemize}
%     %     \item {Made an online carrom game using three.js and WebGL implementing collisions and different camera angles.}
%     % \end{itemize}
%     % \smallskip

%     % \href{https://github.com/shaleengarg/2Dgame_opengl}{\textbf{2D Maze Game}} \hfill \emph{March'16}\\
%     % \emph{Course Project: Dr.PJ Narayanan(Professor \& Director, IIIT-H), Graphics}
%     % \begin{itemize}
%     %     \item {A 3D computer game programmed using OpenGL C++ API\@. The game uses a self-developed physics engine.}
%     % \end{itemize}
%     % \smallskip

%     \href{https://youtu.be/MS8gUX3UMek}{\textbf{NetZero Energy House for Hyderabad}} \hfill \emph{April'16}\\
%     \emph{Course Project: Dr. K.S. Rajan(Associate Prof, IIIT-H), Engineering Systems}
%     \begin{itemize}
%         \item{Served as a \textbf{subgroup leader} in a 5 member sub-team to design the Envelope of the system}
%         \item{Worked with a 25 member team to propose a fully functional NetZero Energy residential house for Hyderabad, India}
%     \end{itemize}
%     \smallskip

%     % \href{https://github.com/shaleengarg/P2Pfileshare}{\textbf{P2P File-share}} \hfill \emph{March'16}\\
%     % \emph{Course Project: Dr. Ganesh Iyer(Visiting Faculty, IIIT-H), Computer Networks}
%     % \begin{itemize}
%     %     \item {File transfer between two clients with TCP and UDP sockets in python.}
%     %     \item {Supports file upload/download with MD5sum checks and indexed searching.}
%     %     \item {It periodically checks for any changes made to the shared folders}
%     % \end{itemize}
%     % \smallskip

%     % \href{https://github.com/shaleengarg/UDP_Pinger}{\textbf{UDP Pinger}} \hfill \emph{April'16}\\
%     % \emph{Course Project: Dr. Ganesh Iyer(Visiting Faculty, IIIT-H), Computer Networks}
%     % \begin{itemize}
%     %     \item {Works as an alternative to the standard ICMP ping.}
%     %     \item {Reports Min, Max and Average Round Trip Time and calculates percentage Packet Loss}
%     % \end{itemize}
%     % \smallskip

%     % \textbf{Image Compression} \hfill \emph{March'16}\\
%     % \emph{Course Project: Dr.Vineet Gandhi(Senior Research Scientist, IIIT-H), Digital Signal Analysis}
%     % \begin{itemize}
%     %     \item {Implemented image compression using discrete cosine transform and quantization on Matlab}
%     % \end{itemize}
%     % \smallskip
% \end{body}


%%%%%%%%%%%%%%%%%%%%%%%%%%%%%%%%%%%%%%%%%%%%%%%%%%%%%%%%%%%%%%%%%%%%%%%%%%%%%%%%
% Relevant Courses
% \smallskip
% \header{Relevant Courses} % (* Pursuing in Monsoon'17)}

% \begin{body}
%     \vspace{4pt}
%     \begin{multicols}{3}
%         Data Structures and Algorithms\\ 
%         Operating Systems\\
%         Formal Methods\\
%         %Digital Signal Analysis\\
%         Complexity and Advanced Algorithms\\
%         Distributed Systems\\
%         Statistical Methods in AI\\
%         Systems and Network Security\\
%         Parallel and Scientific Computing\\
%         Compilers\\
%         %Principles Of Programming- -Languages\\
%         %Advanced Computer Networks\\
%     \end{multicols}

% \end{body}

%%%%%%%%%%%%%%%%%%%%%%%%%%%%%%%%%%%%%%%%%%%%%%%%%%%%%%%%%%%%%%%%%%%%%%%%%%%%%%%%
% Skills
\smallskip
\header{Computer Skills}

\begin{body}
    \vspace{14pt}
    \emph{\textbf{Languages:}}{} C, C++, CUDA, Java\\
    \emph{\textbf{Scripting Languages:}}{} Python, Bash\\
    \emph{\textbf{Other Tools/Languages:}}{} \LaTeX\\
    \emph{\textbf{Platforms:}}{} Linux\\
    \emph{\textbf{Hardware:}}{} Arduino, ESP8266(wifi module)
\end{body}

%%%%%%%%%%%%%%%%%%%%%%%%%%%%%%%%%%%%%%%%%%%%%%%%%%%%%%%%%%%%%%%%%%%%%%%%%%%%%%%%
% Other Achievements
%\smallskip
%\header{Positions and Community Service}

%\begin{body}
%    \vspace{14pt}
%    \begin{itemize}
%        \item{Chief Election Commissioner \& Member of the \textbf{Student's Parliament} for academic year 2018-19, IIIT}
%        \item{\textbf{Research Assistant} with Dr. Kishore Kothapalli, academic year 2018-19, CSTAR, IIIT}
%\item{\textbf{Teaching Assistant} for Distributed Systems(CSE431), Monsoon 2017 \& Spring 2018, IIIT}
%        \item{\textbf{Server Administrator} at Center for Security, Theory and Algorithmic Research (CSTAR), IIIT}
%        \item{\textbf{Volunteer} to give weekly science lessons to the local underprivileged children at Ashakiran(Ray of Hope), IIIT}
%        \item{Runners-Up in Mens Doubles \textbf{Table-Tennis} tournament in the annual sports meet 2016, IIIT}
%        \item{Second Runners-Up in Mens Singles \textbf{Table-Tennis} tournament in the annual sports meet 2016, IIIT}
        % \item{Second Runners-Up in under-18 Mens Singles \textbf{Table-Tennis} in the annual sports meet at St. George's College, Mussoorie}
%    \end{itemize}
%\end{body}

\end{document}